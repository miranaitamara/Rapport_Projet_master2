\subsection{Quotient de Rayleigh}

Les (in)stabilités des systèmes dynamiques est l'un des problèmes de valeur propre qui sont omniprésents, et le quotient de rayleigh a pour but d'obtenir des estimations sur ces valeurs propres de plus en plus précises.


$\definition$ \textit{Le quotient de Rayleigh d'un problème de valeur propre généralisé $Ax = \lambda Bx$ est défini comme 
$$ RQ(x)= \frac{x^{T}Ax}{x^{T}Bx}$$
 où $A \succeq 0$ et $B \succ 0$.}

\textit{On peut voir que si $x$ est un vecteur propre telque $x \in \reels^{d} \backslash \{0\}$, $RQ (x)$ est la valeur propre correspondante avec $ \lambda_{min} \leq RQ(x) \leq \lambda_{max}$ où $\lambda_{min}$ et $\lambda_{max}$ sont les valeurs propres extrêmes de A.}

 
\textit{La premi\`ere classe de probl\`emes de valeurs propres est celle pour l'aquelle B est également défini positif.Un tel problème de valeur propre est équivalent \`a un probl\`eme de valeur propre symétrique $B^{-1 / 2}AB^{-1/2}y=\lambda x$ \cite{Ren-Cang}.}








